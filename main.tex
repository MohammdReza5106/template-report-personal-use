\documentclass[12pt]{article}
\input{etc/cmd}

\begin{document}


\fontsize{12pt}{14pt}\selectfont

\begin{minipage}{0.1\textwidth}
\includegraphics[width=2cm]{figs/logo.png}
\end{minipage}%
\hfill%
\begin{minipage}{0.6\textwidth}\centering
\fontsize{10pt}{10pt}\selectfont
به‌نام خدا \\
پروژه کوتاه \\
آنالیز عددی پیشرفته \\
محمدرضا غلامی \\
\vspace{0.25cm}
\begingroup
\fontsize{8pt}{8pt}\selectfont
دانشگاه صنعتی امیرکبیر، دانشکده مهندسی مکانیک \\
تیر ۱۴۰۲ \\
\endgroup
\end{minipage}%
\hfill%
\begin{minipage}{0.1\textwidth}
\includegraphics[width=1.8cm]{figs/images.png}
\end{minipage}

\vspace{0.5cm}

\noindent\rule{\textwidth}{1pt}

\begin{abstract}
\noindent
این پروژه برای الانیلی عددی
\end{abstract}

\section{شرح مسئله}

معادله غیر خطی تیر اویلر-برنولی با مقطع یکنواخت، در حالت استاتیکی را بصورت زیر در نظر بگیرید:
\begin{equation}
	E I \frac{\partial^4 w}{\partial x^4}-\frac{3}{2} E A\left(\frac{\partial w}{\partial x}\right)^2\left(\frac{\partial^2 w}{\partial x^2}\right)=q(x)
\end{equation}
که در آن$E$ مدول الاستیسیته، $I$ ممان اینرسی مقطع تیر، $w(x)$ خیز تیر،$q(x)$ بار گسترده، $A$ مساحت مقطع و $L$ طول تیر میباشد.
\\
\begin{itemize}
  \item هندسه تیر: مقطع تیر مستطیلی به ابعاد $0.1m * 0.2m$ (ارتفاع*عرض) و طول تیر $2m$ 
  \item جنس تیر: آلومینیوم، $E = 70GPa$
  \item نیرو ها: $q(x)=10^4 * a\left(\frac{x}{L}\right)^b N / m$
\end{itemize}
از آن جایی که شماره دانشجویی من ۴۰۰۱۲۶۰۳۴  هست پس طبق فرض مسيله تکیه گاه های مسيله من یک سر گیر دار و یک سر تکیه گاه ساده در نظر گرفته می‌شود.
\begin{itemize}
	\item $a=3$
	\item $b=4$
\end{itemize}
	
\section{ روش حل:}
\subsection{معرفی روش های حل:}
\subsubsection{روش \lr{Method of Adjoints}}

%code space
\begin{latin}
\begin{verbatim}
print("Hello, world!")
the new command 
\end{verbatim}
\end{latin}

\subsubsection{روش \lr{GDQ}}

%code space
\begin{latin}
\begin{verbatim}
print("Hello, world!")
the new command 
\end{verbatim}
\end{latin}

\end{document}